\documentclass{beamer}

\usetheme{Berlin}

\usecolortheme{dolphin}

\usepackage{amsmath}
\usepackage{graphicx}
\usepackage{setspace}
\usepackage{animate}

\title{Pi may be a normal number}
\author{Conor McCoid}
\institute{University of Geneva}
\date{03.14.19}

\begin{document}

\frame{\titlepage}

\begin{frame}
\frametitle{The Conjecture}
\begin{block}{Conjecture}
$\pi$ is a normal number.
\end{block}
\end{frame}

\begin{frame}
\frametitle{So what is a normal number?}
\begin{block}{Definition}
A real number (necessarily irrational) is NORMAL in base b if all whole numbers in base b are distributed uniformly in its infinite sequence of digits.
\end{block}
\end{frame}

\begin{frame}
\frametitle{An illustration}
Consider a number x that is normal in base 2.
The odds of a random digit of x being 0 are 50\%, as are the odds of said digit being 1.
Moreover, the odds of a random pair of digits being 00 are 1 in $2^2$, or 25\%, as are the odds of said pair being 01, 10 or 11.

~

Generally, the odds of a random block of $n$ digits being a given whole number of $n$ digits are 1 in $b^n$, where $b$ is the base.
\end{frame}

\begin{frame}
\frametitle{Some examples}
In base 10 (and possibly all bases?):
\begin{itemize}
\item 0.123456789101112...
\item 0.23571113171923...
\item 0.149162536496481100...
\end{itemize}
\end{frame}

\begin{frame}
\frametitle{The history of normal numbers}
\begin{description}
\item[\cite{borel1909probabilites}:] almost all numbers are normal (the non-normal numbers constitute a Lebesgue measure zero set)
\item[\cite{sierpinski1917demonstration}:] one can specify a normal number
\item[\cite{becher2002example}:] there exist computable numbers that are normal in every base
\end{description}
\end{frame}

\begin{frame}
\frametitle{Something everyone knows but no one can prove}
\begin{block}{Conjecture}
Every irrational algebraic number is normal.
\end{block}

It should be noted that no irrational algebraic number has been proven to be normal.

Likewise, no irrational algebraic number has been proven to be non-normal.
\end{frame}

\begin{frame}
\frametitle{Why do we think $\pi$ is normal?}
\begin{description}
\item[\cite{bailey2012empirical}] Statistical calculations on the first four trillion base-16 digits of $\pi$ show it is almost certainly normal in base 16.
\item[\cite{artacho2012walking}] Graphical representations, such as \href{http://gigapan.org/gigapans/106803}{\color{blue}{this one of 100 billion base-4 digits of $\pi$}}, show similarities with pseudorandom walks.
\end{description}
\end{frame}

\begin{frame}
\frametitle{A fun exercise}
Prove that any positive real number is the product of two normal numbers.
\end{frame}

\begin{frame}[allowframebreaks]
\frametitle{References}
\bibliographystyle{apalike}
%\bibliography{/home/mccoid/LaTeX/references}
\bibliography{C:/Users/conmc/Documents/LaTeX/references}
\end{frame}

\end{document}\grid
