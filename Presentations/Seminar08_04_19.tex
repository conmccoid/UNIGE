\documentclass{beamer}

\usetheme{Berlin}

\usecolortheme{dolphin}

\usepackage{amsmath}
\usepackage{graphicx}
\usepackage{setspace}
\usepackage{animate}

\newcommand{\abs}[1]{\left \vert #1 \right \vert}
\newcommand{\norm}[1]{\left \Vert #1 \right \Vert}
\newcommand{\order}[1]{\mathcal{O} \left ( #1 \right )}
\newcommand{\set}[1]{\left \{ #1 \right \}}
\newcommand{\Set}[2]{\left \{ #1 \middle \vert #2 \right \}}

\title{Introduction to Spectral Collocation}
\author{Conor McCoid}
\institute{University of Geneva}
\date{April 8th, 2019}

\begin{document}

\frame{\titlepage}

\begin{frame}

\begin{block}{The continuous problem}
$\mathcal{L} u(x) = f(x)$
\end{block}

\begin{description}
\item[$\mathcal{L}$:] Some linear operator acting on the function $u(x)$
\item[$u(x)$:] Some real-valued function (with some regularity) acting on a point $x \in \Omega \subset \mathbb{R}$
\item[$f(x)$:] Some real-valued function (with possibly different regularity than $u(x)$) acting on the same point $x$
\end{description}

\end{frame}

\begin{frame}

\begin{block}{The discrete problem}
$L_N u_N = f_N$
\end{block}

\begin{description}
\item[$L_N$:] Some operator taking $N$ pieces of information from $u_N$ and returning $N$ pieces of information in $f_N$, ie. $L_N : \mathbb{R}^N \rightarrow \mathbb{R}^N$
\item[$u_N$:] Some set of $N$ pieces of information, ie. $u_N \in \mathbb{R}^N$
\item[$f_N$:] Some set of $N$ pieces of information, ie. $f_N \in \mathbb{R}^N$
\end{description}

\end{frame}

\begin{frame}

By the description of the discrete problem $L_N$ is some matrix of size $N \times N$ and $u_N$ and $f_N$ are both vectors of length $N$.
The solution vector $u_N$ is then $u_N = L_N^{-1} f_N$.

~

We want our solution vector $u_N$ to correspond in some way to the solution function of the continuous problem.
That is, we want
\begin{equation*}
\lim_{N \to \infty} u_N \equiv u(x)
\end{equation*}
in some sense.

% At this stage it would be helpful to draw a diagram, showing the discrete spaces inside the continuous ones and how the projection would work
\end{frame}

\begin{frame}

The discrete space may be defined by a set of basis functions, $\set{\phi_k(x)}_{k=1}^N$.
Our approximation $u_N$ then defines a linear combination of these functions:
\begin{equation*}
u_N \equiv \sum_{k=1}^N a_k \phi_k(x) .
\end{equation*}

\end{frame}

\end{document}