%%%%%%% DO NOT MODIFY UNTIL after \fi (line 38) %%%%%%%%%%%%%%%%%%%%%%%%%%%%%%%%%%%%%%%%%%%
\ifx\justbeingincluded\undefined
\documentclass{article}
%\usepackage{epic,curves,amsmath,amssymb}
\usepackage{multirow}
%\usepackage[pdftex]{colortbl}
\usepackage[pdftex,colorlinks]{hyperref}
\usepackage[pdftex]{graphicx}

\def\myAbstract#1{
{\vspace{0.2cm}\flushleft\sffamily\bfseries\Large Abstract\vspace{0.2cm}}
\\
#1}

\def\myTalk#1{
\section*{#1}
}

\def\myMini#1{
{\flushleft\sffamily\bfseries\LARGE #1\vspace{1cm}}
}

\def\myAuthor#1#2#3#4{
{\noindent #1 #2$^*$ \index{#2@#2, #1} \\ #3 \\ #4 \\}}

\def\myCoauthor#1#2#3#4{
{\noindent #1 #2 \index{#2@#2, #1} \\ #3 \\ #4 \\}}

\def\myOrganizer#1#2#3#4{
{\noindent #1 #2 \index{#2@#2, #1} \\ #3 \\ #4 \\}}


\def\myLoc#1 {{\flushleft{\bf Chairman:   #1 \\}}}

\def\myItem#1#2{\item #1\\ #2}

\begin{document}
\fi
%%%%%%% DO NOT MODIFY ABOVE %%%%%%%%%%%%%%%%%%%%%%%%%%%%%%%%%%%%%%%%%%%


\myTalk{Minisymposium MS17: Cycles in Newton-Raphson-accelerated Alternating Schwarz}% insert talk/minisymposium title here

% replace # above with your MS number

\myAuthor{Conor}{McCoid} %presenting author
{Section de Math\'{e}matiques, Universit\'{e} de Gen\`{e}ve}
{conor.mccoid@unige.ch}

\myCoauthor{Martin J.}{Gander}
{Section de Math\'{e}matiques, Universit\'{e} de Gen\`{e}ve}
{martin.gander@unige.ch}


\myAbstract{
% type your abstract here
Newton-Raphson may be used to accelerate Schwarz methods.
While this can speed up convergence, it can also cause the methods to exhibit problems particular to Newton-Raphson.
This talk explores several examples where acceleration of alternating Schwarz by Newton-Raphson leads to period doubling cycles.
We also examine the sufficient conditions for convergence and how these might be used to create more robust algorithms.
}

\vskip .5 in
\noindent *Speaker



%%%%%%%%%%%%%%%%%%%%%%%%%%%%%%%%%%%%%%%%%%%
%%%%%%% DO NOT MODIFY BELOW
\ifx\justbeingincluded\undefined
\end{document}
\fi
